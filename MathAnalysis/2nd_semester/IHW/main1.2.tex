\documentclass{article}

\usepackage[T2A]{fontenc}
\usepackage[utf8]{inputenc}
\usepackage[russian]{babel}
% Set page size and margins
% Replace `letterpaper' with `a4paper' for UK/EU standard size
\usepackage[letterpaper,top=2cm,bottom=2cm,left=3cm,right=3cm,marginparwidth=1.75cm]{geometry}
\usepackage[utf8]{inputenc}

% Useful packages
\usepackage{listings}
\setlength{\jot}{15pt}
\usepackage{amsmath}
\usepackage{graphicx}
\usepackage[colorlinks=true, allcolors=blue]{hyperref}
\usepackage{amsmath, amssymb, amsfonts}
\usepackage{mathtools}        % улучшения для amsmath
\usepackage{geometry}         % настройка полей
\geometry{margin=2.5cm}
\usepackage{enumitem}
\usepackage{pgfplots}
\pgfplotsset{compat=1.18}
\lstset{basicstyle=\ttfamily\small, inputencoding=utf8, extendedchars=true}
\title{Индивидуальное домашнее задание №1 (2 часть)\\Математический анализ\\Вариант 6}
\author{Григорьев Даниил, ИСУ: 465635\\\\ Группа P3116, поток: Мат Ан Прод 11.3}
\date{16 апреля 2025}
\begin{document}
\maketitle
\section*{Вычислить длину кривой, заданной параметрически или в
полярных координатах}
1. $x = sin^4t, y = cos^2t$
\subsection*{Решение}
\begin{center}
\begin{tikzpicture}
\begin{axis}[
    title={Кривая \( x = \sin^4 t \), \( y = \cos^2 t \)},
    xlabel={\( x \)},
    ylabel={\( y \)},
    xmin=0, xmax=1,
    ymin=0, ymax=1,
    grid=major,
    smooth,
    samples=100
]
\addplot [domain=0:pi, blue, thick] ({sin(deg(x))^4}, {cos(deg(x))^2});
\end{axis}
\end{tikzpicture}
\end{center}

Формула длины кривой в параметрическом виде выглядит так:\[
L = \int_{t_1}^{t_2} \sqrt{(\frac{dx}{dt})^2 + (\frac{dy}{dt})^2}dt 
\]

Вычисляем производные:
\[
\frac{dx}{dt} = 4 \sin^3 t \cdot \cos t
\]
\[
\frac{dy}{dt} = -2 \cos t \cdot \sin t
\]
Подставляем в подынтегральное выражение в формуле длины кривой и преобразуем:
\[
\sqrt{\left(\frac{dx}{dt}\right)^2 + \left(\frac{dy}{dt}\right)^2} = \sqrt{16 \sin^6 t \cos^2 t + 4 \cos^2 t \sin^2 t} = \sqrt{4 \cos^2 t \sin^2 t (4 \sin^4 t + 1)}
\]
\[
    \sqrt{4 \cos^2 t \sin^2 t (4 \sin^4 t + 1)} = 2 |\cos t \sin t| \sqrt{4 \sin^4 t + 1}
\]
Заметим, что $x(t) = sin^4t = (sin^2t)^2 \geq 0$, а также $x(t)$ имеет период $\pi$, то есть $x(t)$ симметрично относительно t = $\pi$
Аналогично $y(t)=cos^2t\geq 0$ и y(t) имеет период $\pi$
Если рассмотреть промежуток $t \in [0, \frac{\pi}{2}]$, то $sint \in [0, 1] \Longrightarrow x(t) = sin^4t \in [0,1]; cost\in [1, 0] \Longrightarrow y(t)=cos^2t\in[1, 0]$, значит график кривой идет от точки (0, 1) к (1, 0) в первой четверти
Если рассматривать дальше промежуток $t\in[\frac{\pi}{2}, \pi], sint$ будет положительным, но начнет убывать, а $cost$ станет отрицательным $\Rightarrow y(t)=cos^2t$ будет положительной, но траектория кривой начнет повторяться, так как $sin^2t, cos^2t$ - периодические с периодом $\pi$. То есть полный цикл $x(t), y(t)$ проходит за $t\in[0, \pi]$ (на $t\in[\frac{\pi}{2}, \pi]]$ кривая "возвращается" обратно по пройденному пути), значит графики кривой на таких промежутках симметричны $\Rightarrow$ длины кривых равны (на графике выше как раз наблюдаем такую ситуацию)\\
Тогда получаем такое выражение для длины кривой, принимая $t\in[0, \pi]$:
\[
L = \int_0^{\pi/2} 2\cos t \sin t \sqrt{4 \sin^4 t + 1} \, dt
\]
Сделаем замену \( u = \sin^2 t \), \( du = 2 \sin t \cos t \, dt \):
\[
L = \int_0^1 \sqrt{4u^2 + 1} \, du
\]
Этот интеграл вычисляется таким образом (интеграл от иррациональной функции):
\[
\int_0^{1} \sqrt{4u^2 + 1} \, du = \left. \frac{u}{2} \sqrt{4u^2 + 1} \Big|_0^1+ \frac{1}{4} \ln \left(\sqrt{4u^2 + 1} -2u \right) \right|_0^1 = \frac{\sqrt{5}}{2} + \frac{1}{4} \ln (\sqrt{5} - 2 ) \approx 1,4789
\]


\subsection*{Ответ}
Длина кривой равна:
\[
\boxed{\frac{\sqrt{5}}{2} + \frac{1}{4} \ln (\sqrt{5} - 2 ) \approx 1,4789}
\]
\section*{Задание 2}
Вычислить длину кривой, заданной параметрически или в
полярных координатах
\[
\phi = \frac{\sqrt{r^2-2}}{\sqrt{2}}-arccos\frac{\sqrt{2}}{r}, 2 \leq r \leq3
\]



Кривая задана в полярных координатах как функция \(\varphi = \varphi(r)\), то есть зависимость угла от радиуса.  
Используем формулу длины кривой в полярных координатах, заданной функцией \(\varphi(r)\):

\[
L = \int_{r_1}^{r_2} \sqrt{1 + r^2 \left( \frac{d\varphi}{dr} \right)^2} \, dr
\]

Где:
\(\varphi(r) = \frac{\sqrt{r^2 - 2}}{\sqrt{2}} - \arccos\left( \frac{\sqrt{2}}{r} \right)\) \\
\(r \in [2, 3]\)

\subsection*{Найдём производную \(\frac{d\varphi}{dr}\):}


\[
\frac{d}{dr} \left( \frac{\sqrt{r^2 - 2}}{\sqrt{2}} \right)
= \frac{1}{\sqrt{2}} \cdot \frac{1}{2\sqrt{r^2 - 2}} \cdot 2r
= \frac{r}{\sqrt{2(r^2 - 2)}}
\]

\[
\frac{d}{dr} \left( \arccos\left( \frac{\sqrt{2}}{r} \right) \right)
= \frac{1}{\sqrt{1 - \left( \frac{\sqrt{2}}{r} \right)^2}} \cdot \frac{d}{dr} \left( \frac{\sqrt{2}}{r} \right)
= \frac{1}{\sqrt{1 - \frac{2}{r^2}}} \cdot \left( -\frac{\sqrt{2}}{r^2} \right)
\]

Итак имеем:
\[
\frac{d\varphi}{dr} = \frac{r}{\sqrt{2(r^2 - 2)}} + \frac{\sqrt{2}}{r^2 \sqrt{1 - \frac{2}{r^2}}}
= \frac{r}{\sqrt{2(r^2 - 2)}} + \frac{\sqrt{2}}{r^2 \cdot \sqrt{\frac{r^2 - 2}{r^2}}}
= \frac{r}{\sqrt{2(r^2 - 2)}} + \frac{\sqrt{2}}{\sqrt{r^2 - 2}}
\]

Приведём к общему знаменателю:

\[
\frac{d\varphi}{dr} = \frac{r + 2}{\sqrt{2(r^2 - 2)}}
\]

Теперь подставим это значение в формулу длины кривой, заданной в полярных координатах:

\[
L = \int_2^3 \sqrt{1 + r^2 \left( \frac{r + 2}{\sqrt{2(r^2 - 2)}} \right)^2 } \, dr
= \int_2^3 \sqrt{1 + \frac{r^2 (r + 2)^2}{2(r^2 - 2)} } \, dr 
\]
С помощью метода Симпсона и языка программирования Python я нашел численное значение интеграла:
\begin{lstlisting}[language=Python]
import numpy as np
from scipy.integrate import simpson

r_values = np.linspace(2, 3, 1001)


integrand = np.sqrt(1 + (r_values**2 * (r_values + 2)**2) / (2 * (r_values**2 - 2)))
length = simpson(integrand, r_values)
print(length)
\end{lstlisting}
После запуска кода получаем значение 4.029614169529282
\subsection*{Ответ}
\boxed{\approx4.029614169529282}
\end{document}