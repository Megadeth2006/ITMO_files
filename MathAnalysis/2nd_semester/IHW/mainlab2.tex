\documentclass{article}

\usepackage[T2A]{fontenc}
\usepackage[utf8]{inputenc}
\usepackage[russian]{babel}
% Set page size and margins
% Replace `letterpaper' with `a4paper' for UK/EU standard size
\usepackage[letterpaper,top=2cm,bottom=2cm,left=3cm,right=3cm,marginparwidth=1.75cm]{geometry}
\usepackage[utf8]{inputenc}

% Useful packages
\usepackage{listings}
\setlength{\jot}{15pt}
\usepackage{amsmath}
\usepackage{graphicx}
\usepackage[colorlinks=true, allcolors=blue]{hyperref}
\usepackage{amsmath, amssymb, amsfonts}
\usepackage{mathtools}        % улучшения для amsmath
\usepackage{geometry}         % настройка полей
\geometry{margin=2.5cm}
\usepackage{enumitem}
\usepackage{pgfplots}
\pgfplotsset{compat=1.18}
\lstset{basicstyle=\ttfamily\small, inputencoding=utf8, extendedchars=true}
\title{Лабораторная работа №2\\Математический анализ\\Вариант 1}
\author{Григорьев Даниил, ИСУ: 465635\\\\ Группа P3116, поток: Мат Ан Прод 11.3}
\date{14 апреля 2025}
\begin{document}
\maketitle

\section*{Задание 1}
\subsection*{Длина кривой Безье}
Задача — протянуть оптоволоконный кабель от точки $A$ до точки $C$, огибая точку $K$ и используя наименьшее количество материала (длина кабеля минимальна). Для моделирования кабеля необходимо использовать единственную кривую Безье второго порядка на плоскости, проходящую через все три точки. В процессе решения нужно в явном виде использовать интегральную формулу вычисления длины кривой. Разрешается использовать любые программные пакеты для выполнения алгебраических операций и взятия интегралов, все вычисления следует привести в отчёте с подробным описанием. В ответе должна присутствовать длина провода и координаты опорных точек кривой Безье. Также необходимо продемонстрировать результат графически. Кривая Безье второго порядка на плоскости имеет следующее уравнение:

\[
B(t) = (1 - t)^2 A + 2t(1 - t)B + t^2C,
\]

где $A, B, C \in \mathbb{R}^2$ — опорные точки кривой. $A(0, 0)$; $C(10, 0)$; $K(x, y)$ зависит от варианта.

В моем варианте: K(x, y): K(2, 7)
\subsection*{Решение}
Кривая должна проходить через точку $K(2,7)$ при некотором $t = t_k$:
\[
\begin{cases}
2 = 2t_k(1-t_k)x_b + 10t_k^2 \\
7 = 2t_k(1-t_k)y_b
\end{cases}
\]

Выражаем координаты опорной точки $B$:
\[
x_b = \frac{2 - 10t_k^2}{2t_k(1-t_k)}, \quad y_b = \frac{7}{2t_k(1-t_k)}
\]
Заметим, что если мы подставим в уравнение кривой Безье t = 0, то получим $B(0) = A + 0*B + 0 * C = A$, при t = 1 имеем: $B(1) = 0*A + 0*B + C = C$, значит пределы интегрирования при вычислении длины кривой будут от 0 до 1 (так как A и С - крайние точки графика кривой).
Тогда длина кривой вычисляется по формуле:
\[
L = \int_0^1 \sqrt{(x'(t))^2 + (y'(t))^2} dt
\]
где производные:
\[
x'(t) = 2(1-2t)x_b + 20t, \quad y'(t) = 2(1-2t)y_b
\]
По условию задачи длина кривой должна иметь минимальное значение. Найдем такое $t_k \in (0,1)$, при котором L будет иметь минимальное значение (сделаем это с помощью метода Симпсона)
\begin{lstlisting}[language=Python]
import numpy as np
from scipy.integrate import simpson

def compute_length_simpson(t_k, n_points=1001):
    B_x = (2 - 10 * t_k**2) / (2 * t_k * (1 - t_k))
    B_y = 7 / (2 * t_k * (1 - t_k))
    t = np.linspace(0, 1, n_points)
    x_prime = 2*(1-2*t)*B_x + 20*t
    y_prime = 2*(1-2*t)*B_y
    integrand = np.sqrt(x_prime**2 + y_prime**2)
    return simpson(integrand, t)

t_min, t_max = 0.01, 0.99 
N = 1000                  
n_int_points = 1001      

t_k_values = np.linspace(t_min, t_max, N)
lengths = np.zeros(N)

for i, t_k in enumerate(t_k_values):
    lengths[i] = compute_length_simpson(t_k, n_int_points)

optimal_idx = np.argmin(lengths)
optimal_t_k = t_k_values[optimal_idx]
optimal_length = lengths[optimal_idx]

optimal_B_x = (2 - 10*optimal_t_k**2)/(2*optimal_t_k*(1-optimal_t_k))
optimal_B_y = 7/(2*optimal_t_k*(1-optimal_t_k))


print(optimal_t_k:.6f)
print(optimal_length:.6f)
print(optimal_B_x:.6f)
print(optimal_B_y:.6f)
\end{lstlisting}
После запуска программы получаем:\\
Минимальная длина L = 18.686170 \\
Достигается при $t_k$ = 0.455365 \\
В этом случае координаты точки B такие:\\
$x_B$= -0.148334 \\
$y_B$ = 14.112462 \\
\newpage
На этом графике синим цветом продемонстрирован график итоговой кривой:\\
\begin{tikzpicture}[scale=0.4]

  % Координатные оси
  \draw[->] (-2, 0) -- (12, 0) node[right] {$x$};
  \draw[->] (0, -2) -- (0, 16) node[above] {$y$};

  % Деления по x
  \foreach \x in {0, 2, 4, 6, 8, 10}
    \draw (\x, 0.2) -- (\x, -0.2) node[below] {\x};

  % Деления по y
  \foreach \y in {2, 4, 6, 8, 10, 12, 14}
    \draw (0.2, \y) -- (-0.2, \y) node[left] {\y};

  % Опорные точки
  \coordinate (A) at (0, 0);
  \coordinate (B) at (-0.14, 14.11);
  \coordinate (C) at (10, 0);

  % Рисуем кривую Безье
  \draw[thick, blue, domain=0:1, samples=100, smooth, variable=\t] 
    plot ({(1-\t)^2 * 0 + 2*(1-\t)*\t*(-0.14) + \t^2 * 10},
          {(1-\t)^2 * 0 + 2*(1-\t)*\t*14.11 + \t^2 * 0});

  % Отображение точек
  \fill[green] (A) circle (4pt) node[above left] {$A(0,0)$};
  \fill[red] (B) circle (4pt) node[above left] {$B(-0.14,\ 14.11)$};
  \fill[green] (C) circle (4pt) node[above right] {$C(10,0)$};
  \fill[orange] (2, 7) circle (4pt) node[above] {$K(2,7)$};

  % Вспомогательные линии
  \draw[dashed, gray] (A) -- (B) -- (C);

\end{tikzpicture}
\subsection*{Ответ:}
Длина кабеля $\approx$ 18.68617\\
Координаты опорных точек:\\
$x_B$= -0.148334 \\
$y_B$ = 14.112462 \\
\end{document}