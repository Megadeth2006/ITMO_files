\documentclass{article}

\usepackage[T2A]{fontenc}
\usepackage[utf8]{inputenc}
\usepackage[russian]{babel}
% Set page size and margins
% Replace `letterpaper' with `a4paper' for UK/EU standard size
\usepackage[letterpaper,top=2cm,bottom=2cm,left=3cm,right=3cm,marginparwidth=1.75cm]{geometry}
\usepackage[utf8]{inputenc}

% Useful packages
\usepackage{listings}
\setlength{\jot}{15pt}
\usepackage{amsmath}
\usepackage{graphicx}
\usepackage[colorlinks=true, allcolors=blue]{hyperref}
\usepackage{amsmath, amssymb, amsfonts}
\usepackage{mathtools}        % улучшения для amsmath
\usepackage{geometry}         % настройка полей
\geometry{margin=2.5cm}
\usepackage{enumitem}
\usepackage{pgfplots}
\pgfplotsset{compat=1.18}
\lstset{basicstyle=\ttfamily\small, inputencoding=utf8, extendedchars=true}
\title{Индивидуальное домашнее задание №1 (третья часть)\\Математический анализ\\Вариант 16}
\author{Григорьев Даниил, ИСУ: 465635\\\\ Группа P3116, поток: Мат Ан Прод 11.3}
\date{14 апреля 2025}
\begin{document}
\maketitle

\section*{Задание}
Исследовать несобственные интегралы на сходимость (в каждом варианте
четыре интеграла). Если подынтегральная функция является
знакопеременной, требуется исследовать интеграл на абсолютную и
условную сходимости.
\subsection*{a)}\[
\int_1^\infty xe^{-x}sinxdx
\]
Применим признак абсолютной сходимости:
\[
\int_0^{\infty} |xe^{-x} \sin x| \, dx \leq \int_0^{\infty} xe^{-x} \, dx
\]
Поскольку $|\sin x| \leq 1$, и $\int_0^{\infty} xe^{-x} dx = 1 < \infty$, то интеграл абсолютно сходится, а значит и условно сходится
\subsection*{Ответ}
\boxed{\text{Сходится и условно, и абсолютно}}
\subsection*{б)}
\[\int_0^{\infty} \frac{x^{-\frac{1}{2}} \cos x}{1 + x^{-\frac{1}{3}}} \, dx
\]
На $[1, \infty)$ функция ведёт себя как:
\[
\frac{x^{-1/2} \cos x}{1 + x^{-1/3}} \sim \frac{x^{-1/2} \cos x}{x^{-1/3}} = x^{-1/6} \cos x
\]
Проверим абсолютную сходимость:
\[
\int_1^{\infty} \left| x^{-1/6} \cos x \right| dx \leq \int_1^{\infty} x^{-1/6} dx = \infty
\]
Виодно, что не сходится абсолютно. Проверим условную сходимость по признаку Дирихле: 1) $\cos x$ — ограничена и постоянно колеблется\\
2) $f(x) = \frac{x^{-1/2}}{1 + x^{-1/3}}$ — убывает и стремится к нулю.\\

Следовательно, интеграл условно сходится, но не абсолютно
\subsection*{Ответ}
\boxed{\text{Условно сходится, но не абсолютно}}
\subsection*{в)}
\[
\int_0^1 \frac{\ln x}{\sqrt{(1 - x^2)^3}} \, dx
\]
Особенность в точке $x \to 0$ из-за логарифма. Посмотрим, как ведет себя подынтегральное выражение при $x \to 0$:
\[
\ln x \to -\infty, \quad \sqrt{(1 - x^2)^3} \to 1 \Rightarrow \frac{\ln x}{\sqrt{(1 - x^2)^3}} \sim \ln x
\]

Исследуем сходимость:
\[
\int_0^1 |\ln x| \, dx = \int_0^1 -\ln x \, dx = \left. -x \ln x + x \right|_0^1 = 1 < \infty
\]

При $x \to 1$:
\[
1 - x^2 \sim 2(1 - x) \Rightarrow \sqrt{(1 - x^2)^3} \sim \sqrt{8(1 - x)^3} \Rightarrow \frac{1}{(1 - x)^{\frac{3}{2}}}
\]
Значит:
\[
\int_{1 - \varepsilon}^1 \frac{|\ln x|}{(1 - x)^{\frac{3}{2}}} dx
\]
Поскольку логарифм ограничен вблизи $x = 1$, интеграл ведёт себя как:
\[
\int_{1 - \varepsilon}^1 \frac{1}{(1 - x)^{\frac{3}{2}}} dx = \infty
\]

Интеграл расходится при $x \to 1$.
\subsection*{Ответ}
\boxed{\text{Расходится}}
\subsection*{г)}
\[
\int_0^2 \frac{\ln(2 - x/2)}{\sin(x - 2)} \cdot ctg\left(\sqrt{\frac{\pi x}{2}}\right) dx
\]\\

Особенность в точке $x = 2$:\\
1)$\sin(x - 2) \sim x - 2$\\
2)$\ln(2 - x/2) \sim \ln(1 - x/4) \to \ln 0 \to -\infty$\\
3)$\ctg\left(\sqrt{\frac{\pi x}{2}}\right) \to \ctg(\sqrt{\pi})$ — конечная функция\\

Значит, при $x \to 2$:
\[
\frac{\ln(2 - x/2)}{\sin(x - 2)} \sim \frac{\ln(2 - x/2)}{x - 2}
\]
Это аналогично интегралу:
\[
\int_0^\delta \frac{\ln x}{x} dx \Rightarrow \text{расходится}
\]
Значит исходный интеграл расходится при $x \to 2$
\subsection*{Ответ}
\boxed{\text{Расходится}}
\end{document}