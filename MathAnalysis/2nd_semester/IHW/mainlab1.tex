\documentclass{article}

\usepackage[T2A]{fontenc}
\usepackage[utf8]{inputenc}
\usepackage[russian]{babel}
% Set page size and margins
% Replace `letterpaper' with `a4paper' for UK/EU standard size
\usepackage[letterpaper,top=2cm,bottom=2cm,left=3cm,right=3cm,marginparwidth=1.75cm]{geometry}
\usepackage[utf8]{inputenc}

% Useful packages
\usepackage{listings}
\setlength{\jot}{15pt}
\usepackage{amsmath}
\usepackage{graphicx}
\usepackage[colorlinks=true, allcolors=blue]{hyperref}
\usepackage{amsmath, amssymb, amsfonts}
\usepackage{mathtools}        % улучшения для amsmath
\usepackage{geometry}         % настройка полей
\geometry{margin=2.5cm}
\usepackage{enumitem}
\usepackage{pgfplots}
\pgfplotsset{compat=1.18}
\lstset{basicstyle=\ttfamily\small, inputencoding=utf8, extendedchars=true}
\title{Лабораторная работа №1\\Математический анализ\\Вариант 16}
\author{Григорьев Даниил, ИСУ: 465635\\\\ Группа P3116, поток: Мат Ан Прод 11.3}
\date{21 апреля 2025}
\begin{document}
\maketitle

\section*{Задание}
Вычислить приближённо интеграл с погрешностью
$\epsilon = 0,00001$
методами: прямоугольников, трапеций, Симпсона.
Интеграл:\[
\int_\frac{\pi}{4}^\frac{\pi}{2} lnsinx\, dx
\]
Разбиение: \( h = \frac{b - a}{n} \), узлы: \( x_i = a + \left(i + \frac{1}{2} \right) h \), \( i = 0, \dots, n-1 \)

Формула:
\[
I_n = h \sum_{i=0}^{n-1} f(x_i)
\]

Погрешность:
\[
|\Delta| \leq \frac{b - a}{24} h^2 \cdot \sup_{x \in [a,b]} |f''(x)|
\]
\subsection*{Метод прямоугольников}
\begin{lstlisting}[language=Python]
import numpy as np

def f(x):
    return np.log(np.sin(x))

def f2(x):  
    return -1 / (np.sin(x)**2) - 1 / (np.tan(x)**2)


a = np.pi / 4
b = np.pi / 2
eps = 1e-5


def rectangles_method(f, f2, a, b, eps):
    n = 1
    while True:
        h = (b - a) / n
        x = a + h / 2
        total = 0
        for _ in range(n):
            total += f(x) * h
            x += h
        x_vals = np.linspace(a + 1e-6, b - 1e-6, 1000)
        M2 = np.max(np.abs(f2(x_vals)))
        error = (b - a) / 24 * h**2 * M2

        if error < eps:
            return total, n, h, error
        n *= 2
I_rect, n_rect, h_rect, err_rect = rectangles_method(f, f2, a, b, eps)
print(I_rect)
\end{lstlisting}
Выводится строка:\\
-0.08641215676281004 - приближенное численное значение интеграла по методу прямоугольников с погрешностью $\epsilon = 0,00001$

\section*{Метод трапеций}
Узлы: \( x_i = a + i h \), где \( h = \frac{b - a}{n} \), \( i = 0, \dots, n \)

Формула:
\[
I_n = \frac{h}{2} \left(f(x_0) + 2 \sum_{i=1}^{n-1} f(x_i) + f(x_n) \right)
\]

Погрешность:
\[
|\Delta| \leq \frac{b - a}{12} h^2 \cdot \sup_{x \in [a,b]} |f''(x)|
\]
\begin{lstlisting}[language=Python]
import numpy as np

def f(x):
    return np.log(np.sin(x))

def f2(x):  
    return -1 / (np.sin(x)**2) - 1 / (np.tan(x)**2)


a = np.pi / 4
b = np.pi / 2
eps = 1e-5


def trapezia_method(f, f2, a, b, eps):
    n = 1
    while True:
        h = (b - a) / n
        x = np.linspace(a, b, n + 1)
        y = f(x)
        total = h * (y[0] + 2 * np.sum(y[1:-1]) + y[-1]) / 2

        x_vals = np.linspace(a + 1e-6, b - 1e-6, 1000)
        M2 = np.max(np.abs(f2(x_vals)))
        error = (b - a) / 12 * h**2 * M2

        if error < eps:
            return total, n, h, error
        n *= 2
I_trap, n_trap, h_trap, err_trap = trapezia_method(f, f2, a, b, eps)
print(I_trap)
\end{lstlisting}
Выводится строка:\\
-0.08641686294215971 - приближенное численное значение интеграла по методу трапеций с погрешностью $\epsilon = 0,00001$

\section*{Метод Симпсона}
Требует чётное число разбиений \( n = 2m \). Формула:
\[
I_n = \frac{h}{3} \left(f(x_0) + 2 \sum_{j=1}^{m-1} f(x_{2j}) + 4 \sum_{j=1}^{m} f(x_{2j - 1}) + f(x_n)\right)
\]

Погрешность:
\[
|\Delta| \leq \frac{b - a}{180} h^4 \cdot \sup_{x \in [a,b]} |f^{(4)}(x)|
\]

\begin{lstlisting}[language=Python]
import numpy as np

def f(x):
    return np.log(np.sin(x))

def f2(x):  
    return -1 / (np.sin(x)**2) - 1 / (np.tan(x)**2)
def f4(x):  
    sinx = np.sin(x)
    cosx = np.cos(x)
    return (
        6 * cosx**2 / sinx**4 +
        8 * cosx**2 / sinx**2 +
        2 / sinx**2 +
        2 / np.tan(x)**2
    )

a = np.pi / 4
b = np.pi / 2
eps = 1e-5


def simpson_method(f, f4, a, b, eps):
    n = 2
    while True:
        h = (b - a) / n
        x = np.linspace(a, b, n + 1)
        y = f(x)
        total = h/3 * (y[0] + 4 * np.sum(y[1:n:2]) + 2 * np.sum(y[2:n-1:2]) + y[n])

        x_vals = np.linspace(a + 1e-4, b - 1e-4, 1000)
        M4 = np.max(np.abs(f4(x_vals)))
        error = (b - a) / 180 * h**4 * M4

        if error < eps:
            return total, n, h, error
        n *= 2
I_simp, n_simp, h_simp, err_simp = simpson_method(f, f4, a, b, eps)
print(I_simp)
\end{lstlisting}
Выводится число:\\
-0.08641385377900195 - численное приближенное значение интеграла вычисленное методом Симпсона с погрешностью $\epsilon = 0,00001$
\end{document}